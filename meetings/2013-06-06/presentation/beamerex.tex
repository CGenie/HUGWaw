\documentclass{beamer}


\usepackage{amsmath,amssymb,amsfonts}
\usepackage{hyperref}
\usepackage[english]{babel}
\usepackage{times}
\usepackage[T1]{fontenc}

\mode<presentation>

  \usetheme{Warsaw}
  

  \setbeamercovered{transparent}

\def\R{{\mathbb R}}
\def\C{{\mathbb C}}
\def\bS{{\mathbb S}}

\title[] % (optional, use only with long paper titles)
{Acceleration bundles\\ on Banach and Fr\'{e}chet manifolds}

\subtitle {\em JGP Editorial Board Scientific Meeting\\
In commemoration of Andr\'{e} Lichnerowicz\\ 27-29 June 2006, International
School for Advanced Studies, Miramare, Trieste Italy  } % (optional)

\author[] % (optional, use only with lots of authors)
{C.T.J. Dodson$^1$ and G.A. Galanis$^2$}

\institute[ ] % (optional, but mostly needed)
{$^1$School of Mathematics,
  University of Manchester\\ $^2$Section of Mathematics, Naval Academy of Greece}

\date[ ] % (optional)
{}

\subject{Talks}
% This is only inserted into the PDF information catalog. Can be left
% out.



% If you have a file called "university-logo-filename.xxx", where xxx
% is a graphic format that can be processed by latex or pdflatex,
% resp., then you can add a logo as follows:

 %\pgfdeclareimage[height=0.5cm]{university-logo}{university-logo-filename}
 %\logo{\pgfuseimage{university-logo}}



% Delete this, if you do not want the table of contents to pop up at
% the beginning of each subsection:
%\AtBeginSubsection[]
%{
%  \begin{frame}<beamer>
%    \frametitle{Outline}
%    \tableofcontents[currentsection,currentsubsection]
%  \end{frame}
%}


% If you wish to uncover everything in a step-wise fashion, uncomment
% the following command:

%\beamerdefaultoverlayspecification{<+->}


\begin{document}
\section{Title}
\begin{frame}
  \titlepage
\end{frame}

\section{Abstract}
\begin{frame}
{\bf Abstract} The second order tangent bundle $T^{2}M$ of a
smooth manifold $M$ consists of the equivalence classes of curves
on $M$ that agree up to their
acceleration. Dodson and Radivoiovici~\cite{Dod} showed that in the case of a finite $n$%
-dimensional manifold $M$, $T^{2}M$ becomes a vector bundle over
$M$ if and only if $M$ is endowed with a linear connection.

\vspace{0.5cm}
We have extended this result to $M$ modeled on an arbitrary
Banach space and more generally to those Fr\'{e}chet manifolds
which can be obtained as projective limits of Banach manifolds.
Various structural properties have been deduced.
\end{frame}
\section{Introduction}{\bf Introduction}\\
It is an honour to be able to contribute to the memory of Andr\'{e} Lichnerowicz through
this meeting. The point of contact with his legacy is our attempt to follow his
tradition of developing global differential geometric structures that help to model
real phenomena. This characteristic of Lichnerowicz's work Bourguignon
has highlighted in the 1999 memorial article~\cite{Berger}
and it is evident also through the 1976 volume in honour of Lichnerowicz's 60$^{th}$
birthday~\cite{Cahen}.

\vspace{0.5cm}
Lichnerowicz himself firmly placed in the context of
quantum and statistical mechanics his work on deformations of algebras of
smooth functions on a smooth Banach manifold~\cite{AL}.

\newpage

Our constructions have provided in the Fr\'{e}chet manifold case a
suitable principal bundle of frames $F^{2}M$ for the second tangent bundle
$T^{2}M,$ which is a vector bundle in the presence of a linear connection.
Then $T^{2}M$ is associated with ${F}^{2}M$ and a one to one
correspondence between their connections is provided.

\vspace{0.5cm}
Fr\'{e}chet spaces of sections arise naturally as
configurations of a physical field and evolution equations
naturally involve second order operators.

\vspace{0.5cm}
We mention first some areas of potential
application for our results.

\newpage

The moduli space of inequivalent configurations of a physical field is the quotient of the
infinite-dimensional configuration space $\mathcal{X}$ by the appropriate
symmetry gauge group.

\vspace{0.5cm}
Typically, $\mathcal{X}$ is modelled on a Frech\'{e}t
space of smooth sections of a vector bundle over a closed manifold and is a
Hilbert Lie group.


\vspace{0.5cm}
Inverse limit Hilbert manifolds and inverse limit Hilbert
groups, introduced by Omori~\cite{Om,omori}, provide an appropriate setting
for the study of the Yang-Mills and Seiberg-Witten field equations.


\newpage
Let $M$ be a finite-dimensional path-connected Riemannian manifold.
The free loop space of all smooth maps from the circle group $S^{1}$ to $M$ is a
Fr\'{e}chet manifold $\Lambda M,$ cf.
Manoharan~\cite{manoharan98,manoharan}.

\vspace{0.3cm}
A string structure is defined as a lifting of the structure
group to an $S^{1}$-central extension of the loop group. Suppose that
$\widetilde{G}\rightarrow \widetilde{P}\rightarrow X$ is a lifting of a
principal Fr\'{e}chet bundle $G\rightarrow P\rightarrow X$ over a Fr\'{e}%
chet manifold $X$ and further that $S^{1}\rightarrow \widetilde{G}%
\rightarrow G$ is an $S^{1}$-central extension of $G.$ Manoharan showed that
every connection on the principal bundle $G\rightarrow P\rightarrow X$
together with a $\widetilde{G}$-invariant connection on $S^{1}\rightarrow
\widetilde{P}\rightarrow P$ defines a connection on $\widetilde{G}%
\rightarrow \widetilde{P}\rightarrow X.$

\newpage

The group $\mathcal{D}$ of orientation preserving smooth
diffeomorphisms of a compact manifold $M$ is homeomorphic to the product of
the group of volume preserving diffeomorphisms $\mathcal{D}_{\mu }$, of a
volume element $\mu $ on $M$, times the set $\mathcal{V}$ of all volumes $%
v>0 $ with $\int v=\int \mu $. In this case, $\mathcal{D}_{\mu }$ can be
realized as a projective limit of Hilbert-modelled manifolds (see Omori~\cite%
{Om,omori}) and forms the appropriate framework for the study of
hydrodynamics of an incompressible fluid.

\vspace{0.5cm}

Moreover, there is a close relationship between geodesics on $\mathcal{D}%
_{\mu }$ and the classical Euler equations for a perfect fluid. Namely, if $%
\eta _{t}\in \mathcal{D}_{\mu }$ is a geodesic of $\mathcal{D}_{\mu }$ as
above and $v_{t}=d\eta _{t}/dt$ the velocity, then the vector field $%
u_{t}=v_{t}\circ \eta _{t}^{-1}$ of $M$ is a solution to the classical Euler
equations.

\newpage
The space $J^{\infty }E$ of infinite jets of the sections of a Banach
modelled vector bundle $E$ can be realized as the projective limit of the
finite corresponding jets $\{J^{k}E\}_{k\in \mathbb{N}}$.


\vspace{0.5cm}
This approach
makes possible the definition of a Fr\'{e}chet modelled vector bundle on
$J^{\infty }E$ and thus the use of the latter for the description of
Lagrangians and source equations as certain types of differential forms, cf.
Galanis~\cite{Gal2}, Takens~\cite{Takens} and Lewis~\cite{lewis}.











\newpage
\section{Preliminaries}{\bf Preliminaries}\\
In the case of a finite $n$%
-dimensional manifold $M$,  if and only if $M$ is endowed with a linear
connection, $T^{2}M$ becomes a
vector bundle over $M$ with structure group the general linear
group $GL(2n;\mathbb{R})$ and, therefore, a $3n$-dimensional
manifold~\cite{Dod}.

\vspace{1cm}
{\bf Banach case}\\
Consider a manifold $M$ modeled on an arbitrarily chosen
Banach space $\mathbb{E}$. Using the Vilms~\cite{Vi} point of view
for connections on infinite dimensional vector bundles and a new
formalism, we prove that $T^{2}M$ can be thought of as a Banach
vector bundle over $M$ with structure group $GL(\mathbb{E}\times
\mathbb{E})$ if and only if $M$ admits a linear connection.

\newpage
Let $M$ be a $C^{\infty }-$manifold modeled on a Banach space
$\mathbb{E}$ and atlas $\{(U_{\alpha },\psi _{\alpha })\}_{\alpha \in
I}$. This gives atlas
$\{(\pi _{M}^{-1}(U_{\alpha }),\Psi
_{\alpha })\}_{\alpha \in I}$ for the tangent bundle $TM$ of $M$ with%
\begin{equation*}
\Psi _{\alpha }:\pi _{M}^{-1}(U_{\alpha })\longrightarrow \psi
_{\alpha }(U_{\alpha })\times \mathbb{E}:[c,x]\longmapsto (\psi
_{\alpha }(x),(\psi _{\alpha }\circ c)^{\prime }(0)),
\end{equation*}%
where $[c,x]$ is the equivalence class of smooth curves
$c$ of $M$ with $c(0)=x$ and $(\psi _{\alpha }\circ c)^{\prime
}(0)=[d(\psi _{\alpha }\circ c)(0)](1)$. The
trivializing system of $T(TM)$ is
denoted by
$$\{(\pi _{TM}^{-1}(\pi _{M}^{-1}(U_{\alpha })),\widetilde{\Psi }%
_{\alpha })\}_{\alpha \in I}.$$

\newpage
A connection on $M$ is a vector bundle morphism:%
\begin{equation*}
D:T(TM)\longrightarrow TM
\end{equation*}%
with smooth mappings $\omega _{\alpha
}:\psi
_{\alpha }(U_{\alpha })\times \mathbb{E}\rightarrow \mathcal{L}(\mathbb{E},%
\mathbb{E)}$ defined by the local forms of D:%
\begin{equation*}
D_{\alpha }:\psi _{\alpha }(U_{\alpha })\times \mathbb{E}\times \mathbb{E}%
\times \mathbb{E}\rightarrow \psi _{\alpha }(U_{\alpha })\times
\mathbb{E}
\end{equation*}%
with $D_{\alpha }:=\Psi _{\alpha }\circ D\circ (\widetilde{\Psi
}_{\alpha })^{-1},$ $\alpha \in I,$ via the relation
\begin{equation*}
D_{\alpha }(y,u,v,w)=(y,w+\omega _{\alpha }(y,u)\cdot v).
\end{equation*}%
$D$ is  linear if and only if $%
\{\omega _{\alpha }\}_{\alpha \in I}$ are linear in
the second variable.

Such a connection $D$ is characterized by the
Christoffel
symbols $\{\Gamma _{\alpha }\}_{\alpha \in I}$ ,  smooth mappings%
\begin{equation*}
\Gamma _{\alpha }:\psi _{\alpha }(U_{\alpha })\longrightarrow \mathcal{L}(%
\mathbb{E},\mathcal{L}(\mathbb{E},\mathbb{E}))
\end{equation*}%
defined by $\Gamma _{\alpha }(y)[u]=\omega _{\alpha }(y,u)$,
$(y,u)\in \psi _{\alpha }(U_{\alpha })\times \mathbb{E}$.
 On chart overlaps:
\begin{eqnarray*}
\Gamma _{\alpha }(\sigma _{\alpha \beta }(y))(d\sigma _{\alpha
\beta }(y)(u))[d(\sigma _{\alpha \beta }(y))(v)]+(d^{2}\sigma
_{\alpha \beta
}(y)(v))(u) \\
=d\sigma _{\alpha \beta }(y)((\Gamma _{\beta }(y)(u))(v)),
\end{eqnarray*}
for all $(y,u,v)\in \psi _{\alpha }(U_{\alpha }\cap U_{\beta
})\times \mathbb{E}\times \mathbb{E}.$ Here $d$, $d^{2}$ stand for
the first and the second differential and by
$\sigma _{\alpha \beta }=\psi _{\alpha }\circ \psi _{\beta }^{-1}$ of $%
\mathbb{E}$.
\newpage
{\bf Projective system of Banach manifolds}\\
Let $\{M^{i};\varphi ^{ji}\}_{i,j\in \mathbb{N}}$ be a projective
system of Banach manifolds modeled on the Banach spaces
$\{\mathbb{E}^{i}\}$ respectively. We assume that

\textbf{(i)} the models form also a projective limit
$\mathbb{F}=\varprojlim \mathbb{E}^{i}$,

\textbf{(ii) }for each $x=(x^{i})\in M$ there exists a projective
system of local charts $\{(U^{i},\psi ^{i})\}_{i\in \mathbb{N}}$
such that $x^{i}\in U^{i}$ and the corresponding limit
$\varprojlim U^{i}$ is open in $M$.

Then the projective limit $M=\varprojlim M^{i}$ can be endowed with a Fr\'{e}%
chet manifold structure modeled on $\mathbb{F}$ via the charts $%
\{(\varprojlim U^{i},\varprojlim \psi ^{i})\}$. Moreover, the
tangent bundle $TM$ of $M$ is also endowed with a Fr\'{e}chet
manifold structure of the same type modeled on $\mathbb{F}\times
\mathbb{F}$.

\newpage
The local structure now is defined by the projective
limits of the differentials of $\{\psi ^{i}\}$ and $TM$ turns out
to be an isomorph of $\varprojlim TM^{i}$.

\vspace{0.5cm}
Here we adopt the
definition of Leslie~\cite{LE1},~\cite{LE2} for the
differentiability of mappings between Fr\'{e}chet spaces. However,
the differentiability proposed by Kriegl and Michor~\cite{Mich}
is also suited to our study.

\newpage

\section{$T^2M$ for infinite dimensional Banach manifolds}
{\bf $T^2M$ for infinite dimensional Banach manifolds}\\

\vspace{0.5cm}

Let $M$ be a smooth manifold modeled on the infinite dimensional Banach
space $\mathbb{E}$ and $\{(U_{\alpha },\psi _{\alpha })\}_{\alpha \in I}$ a
corresponding atlas.

\vspace{0.5cm}
For each $x\in M$ we define an equivalence
relation $\approx _{x}$ on
$$C_{x}=\{f:(-\varepsilon ,\varepsilon )\rightarrow M \ | \ f
{\rm smooth \ and} \ f(0)=x, \varepsilon >0\}:$$
\begin{equation*}
f\approx _{x}g\Leftrightarrow f^{^{\prime }}(0)=g^{\prime }(0)\text{ and }%
f^{\prime \prime }(0)=g^{\prime \prime }(0).
\end{equation*}%

\newpage
{\bf Definition}\\
We define the {\em tangent space of order two} of $M$ at the point $x$ to
be the quotient $T_{x}^{2}M=C_{x}/\approx _{x}$ and the {\em tangent
bundle of order two} of $M$ the union of all tangent spaces of order 2: $%
T^{2}M:=\underset{x\in M}{\cup }T_{x}^{2}M$.


Of course, $T_{x}^{2}M$ is a
topological vector space isomorphic to $\mathbb{E}\times \mathbb{E}$ via the
bijection
\begin{equation*}
T_{x}^{2}M\overset{\simeq }{\longleftrightarrow }\mathbb{E}\times \mathbb{E}%
:[f,x]_{2}\longmapsto ((\psi _{\alpha }\circ f)^{\prime }(0),(\psi _{\alpha
}\circ f)^{\prime \prime }(0)),
\end{equation*}%
where $[f,x]_{2}$ stands for the equivalence class of $f$ with respect to $%
\approx _{x}$. However, this structure depends on the choice of the chart $%
(U_{\alpha },\psi _{\alpha })$, hence a definition of a vector bundle
structure on $T^{2}M$ cannot be achieved. With a linear connection we solve this problem.

\vspace{0.5cm}

{\bf Theorem}\\
\label{T2Mvb} Given a linear connection $D$ on $M$,
then $T^{2}M$ becomes a Banach vector bundle with structure
group the general linear group $GL(\mathbb{E}\times \mathbb{E)}$.
Moreover, $T^{2}M$ is isomorphic to $TM\times TM$ since both
bundles are characterized by the same cocycle $\{(d\sigma _{\alpha \beta
}\circ \psi _{\beta })\times (d\sigma _{\alpha \beta }\circ \psi _{\beta
})\}_{\alpha ,\beta \in I}$ of transition functions.
\vspace{0.5cm}

We have also a converse\\

{\bf Theorem}\\
Let $M$ be a smooth manifold modeled on the Banach space $%
\mathbb{E}$. If the second order tangent bundle $T^{2}M$ of $M$ admits a
vector bundle structure, with fibres of type $\mathbb{E}\times \mathbb{E}$,
isomorphic to the product of vector bundles $TM\times TM$, then a linear
connection can be defined on $M$.

\newpage
\section{Conjugacy classes for Banach bundles $T^2M$}
{\bf Conjugacy classes for Banach bundles $T^2M$}\\

\vspace{0.5cm}
Using results of Vassiliou~\cite{Vass},
Dodson, Galanis and Vassiliou~\cite{Dod-Gal-Vass} investigated the classification of
the vector bundle structures induced on $T^2M$ by a linear connection on $M$ via
the conjugacy classes of second order differentials.

\vspace{0.5cm}

Given smooth $g:M\rightarrow N$ between Banach manifolds with linear connections,
$(M,\nabla _{M})$ and $(N,\nabla _{N}),$ the second order differential is well-defined by

\begin{equation*}
T^{2}g:T^{2}M\longrightarrow T^{2}N:[(c,x)]_{2}\mapsto \lbrack
(g\circ c,g(x))]_{2}.
\end{equation*}

\newpage
$\nabla _{M}$ and $\nabla _{N}$ are called $g$\emph{-conjugate}
 if they commute with the differentials of $g:(M,\nabla _{M} )\rightarrow (N,\nabla _{N}):$
\begin{equation*}\label{eq1}
Tg\circ \nabla _{M}=\nabla _{N}\circ T(Tg).
\end{equation*}%
The local expression of the latter is the following:
\begin{equation*}\label{eq2}
 \begin{gathered}
   DG(\phi _{\alpha }(x))(\Gamma _{\alpha }^{M}(\phi _{\alpha
   }(x))(u)(u))=\\
 \Gamma _{\beta }^{N}(G(\phi _{\alpha }(x)))(DG(\phi _{\alpha
}(x))(u))(DG(\phi _{\alpha }(x))(u))+ \\
+ D(DG)((\phi _{\alpha
}(x))(u,u),
   \end{gathered}
\end{equation*}%
for every $(x,u)\in U_{\alpha }\times \mathbb{E}.$

\newpage
{\bf Examples}\\
\textbf{1.} In the case of a constant map $g$, the condition
collapses to a trivial identification of zero
quantities, since the local expression $G $ is constant.
As a result, all linear connections are conjugate through constant
maps.

\textbf{2. }If we consider the identity map $g=id_{M}$, then
\begin{gather*}
D\phi _{\beta \alpha }(\phi _{\alpha }(x))(\Gamma _{\alpha
}^{M}(\phi _{\alpha }(x))(u)(u))=\\[1pt]
 \Gamma _{\beta }^{N}(\phi
_{\beta }(x))(D\phi _{\beta \alpha }(\phi _{\alpha }(x))(u))(D\phi
_{\beta \alpha }(\phi _{\alpha }(x))(u))+D^{2}\phi _{\beta \alpha
}(\phi _{\alpha }(x))(u,u). \notag
\end{gather*}%
The latter is equivalent to the chart overlap compatibility condition satisfied
by the Christoffel symbols of a connection on $M.$ Any
$id_{M}$-conjugate connections have to be equal and the
conjugation relationship in this case reduces to equality.

\newpage

{\bf Theorem}\\
Let $T^{2}M$, $T^{2}N$ be the second order tangent bundles defined
by the pairs $(M,\nabla _{M})$, $(N,\nabla _{N})$, and let
$g:M\rightarrow N$
be a smooth map. If the connections $\nabla _{M}$ and $\nabla _{N}$ are $%
g$-conjugate, then the second order differential
$T^{2}g:T^{2}M\rightarrow T^{2}N$ is a vector bundle morphism~\cite{Dod-Gal-Vass}.

\newpage

{\bf Theorem}\\
Let $\nabla $, $\nabla ^{\prime }$ be two linear connections on
$M$. If $g$
is a diffeomorphism of $M$ such that $\nabla $ and $\nabla ^{\prime }$ are $%
g $-conjugate, then the vector bundle structures on $T^{2}M$, induced by $%
\nabla $ and $\nabla ^{\prime }$, are isomorphic~\cite{Dod-Gal-Vass}. \label{isom}

\vspace{0.5cm}

{\bf Corollary}\\
Up to isomorphism, the elements of the $g$-conjugate equivalence class $%
[(M,\nabla )]_g$ determine the same vector bundle structure on
$T^{2}M$. Consequently, the latter structure depends not only on a
pair $(M,\nabla )$ but also on the entire class $[(M,\nabla )]_g$~\cite{Dod-Gal-Vass}.

\newpage
\section{Fr\'{e}chet manifolds}
{\bf Fr\'{e}chet manifolds}\\
Lewis~\cite{lewis} gives some background material on Fr\'{e}chet spaces and in particular on the
Fr\'{e}chet projective limit of Banach spaces and the
Fr\'{e}chet space of infinite jets---ie Taylor series.

For $M$ modeled on a Fr\'{e}chet
(non-Banach) space $\mathbb{F}$, there are
complications because of the pathological structure of the general linear groups $GL(%
\mathbb{F})$, $GL(\mathbb{F}\times \mathbb{F}),$ which does not even admit
non-trivial topological group structures.

\vspace{0.5cm}
Also, the space of continuous linear mappings
between Fr\'{e}chet spaces does not remain in the same category of
topological vector spaces, and we lack a general solvability
theory of differential equations on $\mathbb{F}.$ These problems are discussed in the
2005 Monastir Summer School Lecture Notes of Neeb~\cite{Neeb}.


\newpage
{\bf Fr\'{e}chet projective limits of Banach manifolds}\\
\vspace{0.2cm}
Restricting ourselves to those Fr\'{e}chet manifolds which can be
obtained as projective limits of Banach manifolds, it
is possible to endow $T^{2}M$ with a vector bundle structure over
$M$ with structure group a new topological (and in a generalized
sense Lie) group which replaces the pathological general linear
group of the fibre type.

\vspace{0.5cm}
This construction is equivalent to the
existence on $M$ of a specific type of linear connection
characterized by a generalized set of Christoffel symbols.



\newpage
Let $M$ be a smooth manifold modeled on the Fr\'{e}chet space $%
\mathbb{F}.$ Taking into account that the latter \emph{always} can be
realized as a projective limit of Banach spaces $\{\mathbb{E}^{i};\rho
^{ji}\}_{i,j\in \mathbb{N}}$ (i.e. $\mathbb{F\cong }\varprojlim \mathbb{E}%
^{i}$) we assume that the manifold itself is obtained as the limit of a
projective system of Banach modeled manifolds $\{M^{i};\varphi
^{ji}\}_{i,j\in \mathbb{N}}$.
Then, we obtain:

\vspace{0.5cm}
{\bf Proposition}\\
The second order tangent bundles $\{T^{2}M^{i}\}_{i\in \mathbb{N}}$ form
also a projective system with limit (set-theoretically) isomorphic to $%
T^{2}M $.\\
\newpage

Next we define a vector bundle structure on $%
T^{2}M $ by means of a certain type of linear connection on $M.$ The
problems concerning the structure group of this bundle
are overcome by the replacement of the pathological $GL(\mathbb{F}\times
\mathbb{F})$ by the new topological (and in a generalized sense smooth Lie)
group:%
\begin{equation*}
\mathcal{H}^{0}(\mathbb{F\times F}):=\{(l^{i})_{i\in \mathbb{N}}\in {%
\prod_{i=1}^{\infty }}GL(\mathbb{E}^{i}\mathbb{\times E}^{i}):\,\varprojlim
l^{i}\,\text{\ exists}\}.
\end{equation*}

\newpage
To be more specific, $\mathcal{H}^{0}(\mathbb{F\times F})$ is a topological
group being isomorphic to the projective limit of the Banach-Lie groups
\begin{equation*}
\mathcal{H}_{i}^{0}(\mathbb{F\times F}):=\{(l^{1},l^{2},...,l^{i})_{i\in
\mathbb{N}}\in {\prod_{k=1}^{i}}GL(\mathbb{E}^{k}\mathbb{\times E}%
^{k}):\,\rho^{jk}\circ l^{j}=l^{k}\circ \rho^{jk}\,\text{\ }
\end{equation*}%
for $k\leq j\leq i.$


\vspace{0.5cm}
On the other hand, it can be considered as a generalized Lie group via its
embedding in the topological vector space $\mathcal{L}(\mathbb{F\times F})$.

\newpage
{\bf Theorem}\\
If a Fr\'{e}chet manifold $M=\varprojlim M^{i}$ is endowed with a linear
connection $D$ that can be also realized as a projective limit of
connections $D=\varprojlim D^{i}$, then $T^{2}M$ is a Fr\'{e}chet vector
bundle over $M$ with structure group $\mathcal{H}^{0}(\mathbb{F\times F}).$\\


\vspace{0.5cm}
Conversely, if $T^{2}M$ is an $\mathcal{H}^{0}(\mathbb{F}\times \mathbb{F})-$Fr\'{e}chet
vector bundle over $M$ isomorphic to $TM\times TM$, then $M$ admits a linear
connection which can be realized as a projective limit of connections.
\vspace{0.5cm}




\newpage
\section{Bibliography}{\bf Bibliography}\\
\begin{thebibliography}{99}

\bibitem{Berger} Marcel Berger, Jean-Pierre Bourguignon, Yvonne Choquet-Bruhat,
Charles-Michel Marle, and Andr\'e Revuz. Andr� Lichnerowicz 1915-1998. {\em Notices
American Mathematical Society} 46, 11, 1388-1396.

\bibitem{Cahen} M. Cahen and M. Flato, Editors. {\bf Differential geometry and relativity.
A volume in honour of Andr� Lichnerowicz on his 60th birthday}.
Mathematical Physics and Applied Mathematics, Vol. 3.
D. Reidel Publishing Co., Dordrecht-Boston, Mass., 1976.

\bibitem{Dod-Gal} C.T.J. Dodson and G.N. Galanis. Second order
tangent bundles of infinite dimensional manifolds. {\em J. Geom.
Phys.}, 52 (2004) 127-136.

\bibitem{Dod-Gal-Vass1} C.T.J. Dodson, G.N. Galanis and E. Vassiliou. A generalized
second-order
frame bundle for Frech\'{e}t manifolds. {\em J. Geom.
Phys.}, 55 (2005) 291-305.


\bibitem{Dod-Gal-Vass} C.T.J. Dodson, G.N. Galanis and E. Vassiliou.
Isomorphism classes for Banach vector bundle structures of
second tangents. {\em Math.~Proc.\ Camb.\ Phil.\
Soc.} 141 (2006) 489-496. \\
 {\tiny \verb+ http://www.maths.manchester.ac.uk/~kd/PREPRINTS/isomt2m.pdf+}

\bibitem{Dod} C.T.J. Dodson and M.S. Radivoiovici.
Tangent and Frame bundles of order two. {\em Analele
stiintifice ale Universitatii ''Al. I. Cuza'',} 28 (1982),
63-71.

\bibitem{Gal2} G. Galanis. Projective Limits of vector bundles.
{\em Portugaliae Mathematica,}  (1998), 11-24.

\bibitem{Mich} A. Kriegl, P. Michor. {\bf The
convenient setting of global analysis}, Mathematical Surveys and Monographs,
53 American Mathematical Society. Providence, RI 1997.


\bibitem{LE1} J.A. Leslie, On a differential structure for the
group of diffeomorphisms. {\em Topology} 46 (1967),
263-271.

\bibitem{LE2} J.A. Leslie. Some Frobenious theorems in Global
Analysis. {\em J. Diff. Geom.} 42 (1968), 279-297.


\bibitem{lewis} A. D. Lewis. The bundle of infinite jets.
Preprint, 2006.
{\tiny \verb+http://penelope.mast.queensu.ca/math949/pdf/infinite-jets.pdf+}


\bibitem{AL} A. Lichnerowicz.
{\bf Deformations and geometric (KMS)-conditions.
Quantum theories and geometry} Les Treilles, 1987, 127-143,
Mathematical Physics Studies, 10,
Kluwer Acad. Publ., Dordrecht, 1988.

\bibitem{manoharan98} P. Manoharan. Characterization for
spaces of sections. {\em Proc. Am. Math. Soc.} {126}, 4, (1998)
1205-1210.

\bibitem{manoharan} P. Manoharan. On the geometry of free
loop spaces. {\em Int. J. Math. Math. Sci.} {30}, 1, (2002) 15-23.

\newpage
\bibitem{Neeb} K-H. Neeb, {\bf Infinite Dimensional Lie Groups},
 2005 Monastir Summer School Lectures, Lecture Notes January 2006.\\
{\tiny \verb+http://wwwbib.mathematik.tu-darmstadt.de/Math-Net/Preprints/Listen/pp06.html+}

\bibitem{Om} H. Omori. On the group of diffeomorphisms on a
compact manifold. {\em Proc. Symp. Pure Appl. Math., XV, Amer. Math.
Soc.} (1970) 167-183.

\bibitem{omori} H. Omori. {\bf Infinite-dimensional Lie groups},
Translations of Mathematical Monographs, {158}, Berlin,
American Mathematical Society (1997).


\bibitem{Takens} F. Takens. A global version of the inverse
problem of the calculus of variations. {\em J. Dif. Geom.}, {14}
(1979), 543-562.

\bibitem{Vass} E. Vassiliou. Transformations of Linear
Connections II. {\em Period. Math. Hung.}, 17, 1
(1986), 1-11.

\bibitem{Vi} J. Vilms. Connections on tangent bundles. {\em J.
Diff. Geom.} 41 (1967), 235-243.
\end{thebibliography}

\end{document}
