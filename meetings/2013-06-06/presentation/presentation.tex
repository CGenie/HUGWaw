\documentclass[landscape]{beamer}

\usepackage{amsmath,amssymb,amsfonts}
\usepackage{graphicx}
\usepackage[T1]{fontenc}
\usepackage[latin1]{inputenc}
\usepackage[polish]{babel}
\usepackage{ifpdf}
\usepackage{color}
\usepackage{verbatim}
\newenvironment{code}{\footnotesize\verbatim}{\endverbatim\normalsize}

\ifpdf
  \DeclareGraphicsRule{*}{mps}{*}{}
\fi

\newcommand{\divergence}{\operatorname{div}}
\newcommand{\D}{\mathcal{D}}
\newcommand{\E}{\mathcal{E}}
\newcommand{\F}{\mathcal{F}}
\newcommand{\R}{\mathbb{R}}
\newcommand{\K}{\mathcal{K}}
\renewcommand{\S}{\mathcal{S}}
\newcommand{\f}{\mathfrak{f}}
\newcommand{\tr}{\operatorname{tr}}
\renewcommand{\skew}{\operatorname{skew}}
\newcommand{\axl}{\operatorname{axl}}
\newcommand{\obciete}{\!\!\mid_{\partial \Omega}}
\newcommand{\hkbr}{{\lbrace h, k \rbrace}}

\newtheorem{thm}{Twierdzenie}
\newtheorem{lem}{Lemat}
\newtheorem{defi}{Definicja}
\newtheorem{rem}{Uwaga}
\newtheorem{cor}{Wniosek}

\definecolor{red}{rgb}{1,0,0}

\mode<presentation>
\usetheme{Warsaw}

\title[Zabawy z Yesod i Fay]{Zabawy z Yesod i Fay}
\author{Przemys\l{}aw Kami\'nski}
\date{6 czerwca 2013 \\ HUG Warsaw}


\begin{document}

\begin{frame}
  \titlepage
\end{frame}

\begin{frame}
  \frametitle{Yesod -- wst\k{e}p}
    \emph {Yesod} (hebr. ``fundament'', ``podstawa'') -- jedna z kabalistycznych
    sefir, \l{}\k{a}cz\k{a}ca \'swiat materialny z duchowym
    \uncover<+->{} \uncover<+->{... oraz Web Framework napisany w Haskellu}
\end{frame}

\begin{frame}
  \frametitle{Yesod -- konfiguracja prostej aplikacji}
  \begin{center}
    \uncover<+->{Szybki wst\k{e}p (dla wersji >= 1.2):}
  \end{center}
  \begin{uncoverenv}<+->
    \texttt{\#> cabal install yesod-platform yesod-bin yesod-test yesod-static
      cabal-dev}
  \end{uncoverenv}

  \begin{uncoverenv}<+->
    \texttt{\#> yesod init \ \ \ \ \# project 'rest' with sqlite}
  \end{uncoverenv}

  \begin{uncoverenv}<+->
    \texttt{\#> cd rest \&\& cabal-dev install \&\& yesod -{}-dev devel}
  \end{uncoverenv}
  
  \begin{uncoverenv}<+->
    \begin{center}
      $\longrightarrow$
      \texttt{http://localhost:3000}
    \end{center}
  \end{uncoverenv}

  \begin{uncoverenv}<+->
    \begin{center}
      Serwer produkcyjny: \\
      \texttt{cabal-dev -fproduction configure \&\& cabal-dev build \&\&
        ./cabal-dev/bin/rest Production}
    \end{center}
  \end{uncoverenv}

  \begin{uncoverenv}<+->
     \texttt{\#> yesod test}
  \end{uncoverenv}
\end{frame}


\begin{frame}
  \frametitle{Yesod -- podstawy}
  \begin{uncoverenv}<+->
    Zmiana w pliku $\Rightarrow$ rekompilacja w locie przez serwer
  \end{uncoverenv}

  \begin{center}
    \uncover<+->{Katalogi:} \\
    \uncover<+->{\texttt{./config/routes}} \\
    \uncover<+->{\texttt{./Handler/}} \\
    \uncover<+->{\texttt{./templates/}}
    \uncover<+->{\texttt{*.\{\alert<.>{hamlet}\only<.>{({\color{blue}{HTML}})\}}}}
    \uncover<+->{\texttt{, \alert<.>{lucius}\only<.>{({\color{blue}{CSS}})\}}}}
    \uncover<+->{\texttt{, \alert<.>{julius}\only<.>{({\color{blue}{JS}})}\}}} \\
    \uncover<+->{\texttt{./config/models}}
  \end{center}

  \begin{uncoverenv}<+->
    \begin{center}
      Zr\'obmy sobie \texttt{git init} i zacommitujmy t\k{a} wersj\k{e}.
    \end{center}
  \end{uncoverenv}
\end{frame}

\begin{frame}
  \frametitle{Yesod -- Echo!}

  Request GET pod \texttt{/echo/\{\alert<+->{str}\}} zwraca HTML z
  \texttt{<h1>\{\alert<.->{str}\}</h1>}.

  \begin{uncoverenv}<+->
    \texttt{\#> yesod add-handler} \\
    \texttt{name: Echo, pattern: /echo/\#String, request: GET}
  \end{uncoverenv}

  \vspace{1em}

  \uncover<+->{\texttt{git status...}} \\
\end{frame}

\begin{frame}
  \frametitle{Yesod -- Echo! c.d.}

  \begin{uncoverenv}<+->
    Edytujemy \texttt{./Handler/Echo.hs}: \\
    \vspace{1em}
    \texttt{getEchoR theText = defaultLayout \\
      \hspace{6em} [whamlet|<h1>\#\{theText\}|]}
  \end{uncoverenv}

  \vspace{1em}

  \uncover<+->{Script injection protection: \\
    \texttt{/echo/\%3Cscript\%3Ealert\%28\%22foo\%22\%29\%3C\%2Fscript\%3E}} \\

  \vspace{1em}

  \uncover<+->{Wszystko dzi\k{e}ki systemowi typ\'ow, kt\'ory konwertuje typ
    \texttt{URL} na \texttt{String} ({\it The web, by its very nature, is not
      type safe. Even the simplest case of distinguishing between an integer and
      string is impossible: all data on the web is transferred as raw bytes,
      evading our best efforts at type safety. Every app writer is left with the
      task of validating all input.}, Yesod Book)}
\end{frame}

\begin{frame}
  \frametitle{Yesod -- Echo! c.d.}

  \begin{uncoverenv}<+->
    Aby zamiast typu \texttt{String} u\.zywa\'c typu
    \texttt{Data.Text}, dodajemy jako ostatni import w pliku
    \texttt{./Foundations.hs}:

    \vspace{1em}

    \texttt{{\color{blue} import} Data.Text}

    \vspace{1em}

    a w pliku \texttt{./config/routes} zamieniamy \texttt{\#String} na
    \texttt{\#Text}. \\
    Modyfikujemy jeszcze syngnatur\k{e} funkcji \texttt{getEchoR} w pliku
    \texttt{./Handler/Echo.hs}.
  \end{uncoverenv}
\end{frame}

\begin{frame}
  \frametitle{Yesod -- system szablon\'ow}

  \begin{uncoverenv}<+->
    \alert<.->{hamlet} -- biblioteka do renderowania HTML z szablon\'ow \\
    \texttt{http://hackage.haskell.org/package/hamlet-1.1.7}
  \end{uncoverenv}

  \vspace{1em}

  \begin{uncoverenv}<+->
    Dodajemy plik \texttt{./templates/echo.hamlet} z zawarto\'sci\k{a}: \\

    \vspace{1em}

    \texttt{<h1>\#\{theText\}}

    \vspace{1em}

    a w pliku \texttt{./Handler/Echo.hs} dajemy \\
    
    \vspace{1em}

    \texttt{getEchoR theText = defaultLayout \$(widgetFile ``echo'')}
  \end{uncoverenv}
\end{frame}

\begin{frame}
  \frametitle{Yesod -- lustro}

  \begin{uncoverenv}<+->
    \texttt{\#> yesod add-handler} \\
    \texttt{name: Mirror, pattern: /mirror, request: GET POST}
  \end{uncoverenv}

  \vspace{1em}

  \uncover<+->{Edytujemy \texttt{./Handler/Mirror.hs}...}
\end{frame}

\begin{frame}
  \frametitle{Yesod -- blog}

  \begin{uncoverenv}<+->
    \texttt{\#> yesod add-handler} \\
    \texttt{name: Blog, pattern: /blog, request: GET POST}
  \end{uncoverenv}

  \begin{uncoverenv}<+->
    \texttt{\#> yesod add-handler} \\
    \texttt{name: Article, pattern: /blog/\#ArticleId, request: GET}
  \end{uncoverenv}

  \vspace{1em}

  \begin{uncoverenv}<+->
    Edytujemy \texttt{./config/models}...
  \end{uncoverenv}
 \uncover<+->{w konsoli z serwerem widzimy komunikat \\
   \texttt{Migrating: CREATE TABLE ``article''(``id'' INTEGER PRIMARY
     KEY,``title'' VARCHAR NOT NULL,``content'' VARCHAR NOT NULL)}}

  \vspace{1em}

  \uncover<+->{Edytujemy \texttt{./Handler/Blog.hs}... \\}
  \uncover<+->{\hspace{2em} \texttt{./templates/articles.hamlet}... \\}
  \uncover<+->{\hspace{2em} \texttt{./templates/articleError.hamlet}... \\}
  \uncover<+->{\hspace{2em} \texttt{./Handler/Article.hs}... \\}
  \uncover<+->{\hspace{2em} \texttt{./templates/article.hamlet}...}
\end{frame}

\begin{frame}
  \frametitle{Hamlet}

  \begin{itemize}
    \uncover<+->{\item sterowany indentacj\k{a}}
    \uncover<+->{$\rightarrow$ bez tag\'ow zamykaj\k{a}cych! \\}
    \uncover<+->{\texttt{<h1>Tytu\l{}} \\ \hspace{2em}}
    \uncover<+->{lub \\
      \texttt{<h1>D\l{}u\.zszy \# \\
              \ \ \ \ tytu\l{}}}
    \uncover<+->{\item nie trzeba domyka\'c tag\'ow, je\'sli nic za nimi nie
      b\k{e}dzie w danej linii: \\
      \texttt{<p \\
              \ \ \ \ <a href=@\{MyHandler\}>My Handler}}
    \uncover<+->{\item Zmienne:}
    \begin{itemize}
      \uncover<+->{\item \texttt{\#\{...\}} -- zwyk\l{}a zmienna \\}
      \only<.>{np.: \\
          \texttt{{\color{blue} import} {\color{green} qualified} Data.Text
            {\color{green} as} T \\
            getTestR = {\color{blue} do} \\
            \hspace{2em} var <- return \$ toHtml \$ T.pack ``<script>'' \\
            defaultLayout [whamlet|Variable: \#\{var\}|]}}
      \uncover<+->{\item \texttt{@\{...\}} -- interpolacja URL (type-safe)}
      \uncover<+->{\item \texttt{\^{}\{...\}} -- wklejanie innych szablon\'ow}
      \uncover<+->{\item \texttt{@?\{...\}} -- URL z parametrami GET}
    \end{itemize}
  \end{itemize}
\end{frame}

\begin{frame}
  \frametitle{Julius, Cassius}

  \uncover<+->{CSS -- \emph{Cassius} (lub \emph{Lucius}) \\}
  \uncover<+->{JS -- \emph{Julius}}
  
  \vspace{1em}

  \uncover<+->{Przyk\l{}ad: \texttt{./Handlers/Shakespeare.hs}...}
\end{frame}

\begin{frame}
  \frametitle{Yesod -- REST + JSON (AESON)}
  
  \begin{uncoverenv}<+->
    \begin{center}
      \texttt{,,Press ENTER to quit''}
      albo
      {\color<1>{blue}{\texttt{./devel.hs}}}
      $\rightarrow$ \texttt{terminateDevel}
    \end{center}
  \end{uncoverenv}

  \uncover<+->{}
\end{frame}


\end{document}
