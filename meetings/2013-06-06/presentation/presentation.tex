\documentclass[landscape]{beamer}

\usepackage{amsmath,amssymb,amsfonts}
\usepackage{graphicx}
\usepackage[T1]{fontenc}
\usepackage[latin1]{inputenc}
\usepackage[polish]{babel}
\usepackage{ifpdf}
\usepackage{color}
\usepackage{verbatim}
\newenvironment{code}{\footnotesize\verbatim}{\endverbatim\normalsize}

\ifpdf
  \DeclareGraphicsRule{*}{mps}{*}{}
\fi

\newcommand{\divergence}{\operatorname{div}}
\newcommand{\D}{\mathcal{D}}
\newcommand{\E}{\mathcal{E}}
\newcommand{\F}{\mathcal{F}}
\newcommand{\R}{\mathbb{R}}
\newcommand{\K}{\mathcal{K}}
\renewcommand{\S}{\mathcal{S}}
\newcommand{\f}{\mathfrak{f}}
\newcommand{\tr}{\operatorname{tr}}
\renewcommand{\skew}{\operatorname{skew}}
\newcommand{\axl}{\operatorname{axl}}
\newcommand{\obciete}{\!\!\mid_{\partial \Omega}}
\newcommand{\hkbr}{{\lbrace h, k \rbrace}}

\newtheorem{thm}{Twierdzenie}
\newtheorem{lem}{Lemat}
\newtheorem{defi}{Definicja}
\newtheorem{rem}{Uwaga}
\newtheorem{cor}{Wniosek}

\definecolor{red}{rgb}{1,0,0}

\mode<presentation>
\usetheme{Warsaw}

\title[Zabawy z Yesod i Fay]{Zabawy z Yesod i Fay}
\author{Przemys\l{}aw Kami\'nski}
\date{6 czerwca 2013 \\ HUG Warsaw}


\begin{document}

\begin{frame}
  \titlepage
\end{frame}

\begin{frame}
  \frametitle{Yesod -- wst\k{e}p}
    \emph {Yesod} (hebr. ,,podstawa'') -- jedna z kabalistycznych sefir,
    \l{}\k{a}cz\k{a}ca \'swiat materialny z duchowym
    \uncover<+->{} \uncover<+->{... oraz Web Framework napisany w Haskellu}
\end{frame}

\begin{frame}
  \frametitle{Yesod -- konfiguracja prostej aplikacji}
  \begin{center}
    \uncover<+->{Szybki wst\k{e}p (dla wersji >= 1.2):}
  \end{center}
  \begin{uncoverenv}<+->
    \verb;\#> cabal install yesod-platform yesod-bin yesod-test yesod-static cabal-dev
  \end{uncoverenv}

  \begin{uncoverenv}<+->
    \verb;\#> yesod init \ \ \ \ \# project 'rest' with sqlite
  \end{uncoverenv}

  \begin{uncoverenv}<+->
    \verb;\#> cd rest \&\& cabal-dev install \&\& yesod -{}-dev devel
  \end{uncoverenv}
  
  \begin{uncoverenv}<+->
    \begin{center}
      $\longrightarrow$
      \verb;http://localhost:3000
    \end{center}
  \end{uncoverenv}

  \begin{uncoverenv}<+->
    \begin{center}
      Serwer produkcyjny: \\
      \verb;cabal-dev -fproduction configure \&\& cabal-dev build \&\& ./cabal-dev/bin/rest Production
    \end{center}
  \end{uncoverenv}

  \begin{uncoverenv}<+->
     \verb;\#> yesod test
  \end{uncoverenv}
\end{frame}


\begin{frame}
  \frametitle{Yesod -- podstawy}
  \begin{uncoverenv}<+->
    Zmiana w pliku $\Rightarrow$ rekompilacja w locie przez serwer
  \end{uncoverenv}

  \begin{center}
    \uncover<+->{Katalogi:} \\
    \uncover<+->{\texttt{./config/routes}} \\
    \uncover<+->{\texttt{./Handler/}} \\
    \uncover<+->{\texttt{./templates/}}
    \uncover<+->{\texttt{*.\{\alert<.>{hamlet}\only<.>{({\color{blue}{HTML}})\}}}}
    \uncover<+->{\texttt{,\alert<.>{lucius}\only<.>{({\color{blue}{CSS}})\}}}}
    \uncover<+->{\texttt{,\alert<.>{julius}\only<.>{({\color{blue}{JS}})}\}}} \\
    \uncover<+->{\texttt{./config/models}}
  \end{center}

  \begin{uncoverenv}<+->
    \begin{center}
      Zr\'obmy sobie \textt{git init} i zacommitujmy t\k{a} wersj\k{e}.
    \end{center}
  \end{uncoverenv}
\end{frame}

\begin{frame}
  \frametitle{Yesod -- Echo!}

  Request GET pod \texttt{/echo/\{\alert<+->{str}\}} zwraca HTML z
  \texttt{<h1>\{\alert<.->{str}\}</h1>}.

  \begin{uncoverenv}<+->
    \texttt{\#> yesod add-handler} \\
    \texttt{name: Echo, pattern: /echo/\#String, request: GET}
  \end{uncoverenv}

  \uncover<+->{\texttt{git status...}}
\end{frame}

\begin{frame}
  \frametitle{Yesod -- poka\.z kotu co masz w \'srodku}
  
  \begin{uncoverenv}<+->
    \begin{center}
      \texttt{,,Press ENTER to quit''}
      albo
      {\color<1>{blue}{\verb;./devel.hs}}
      $\rightarrow$ \texttt{terminateDevel}
    \end{center}
  \end{uncoverenv}

  \uncover<+->{}
\end{frame}


\end{document}
